\section{Техническое задание}
\subsection{Основание для разработки}

Основанием для разработки является задание на выпускную квалификационную работу бакалавра "<Разработка движка \textquotedbl для игр-платформеров">.

\subsection{Цель и назначение разработки}

Основной задачей выпускной квалификационной работы является разработка "движка" игр-платформеров для продвижения их популярности. Данный программный продукт предназначен для демонстрации практических навыков, полученных в течение обучения.

Целью данной разработки является создание движка платформер игры и популяризация этого игрового жанра.

Задачами данной разработки являются:
\begin{itemize}
\item проектирование интерфейса;
\item проектирование игрового сценария;
\item реализация взаимодействия приложения с пользователем;
\item реализация механик игр-платформеров;
\item реализация графики приложения;
\item реализация камеры;
\end{itemize}

\subsection{Требования пользователя к движку}

Движок должен включать в себя:
\begin{itemize}
    \item реализацию основных механик платформера;
    \item реализацию физики платформ и персонажа;
    \item реализация смены сценариев;
    \item реализация передвижения персонажа по уровню;
    \item реализацию камеры.
\end{itemize}



\subsection{Правила игры}
!!!!!!!!!!!!!!Главный герой игры (ГГ) перемещается по уровням собирая различные предметы(усилители) и кольца(очки). Внутри каждого уровня существуют препятствия и враги которых главному герою предстоит преодолеть. Начинается игра с предыстории в виде текста в диалоговой панели, который пользователь должен прочесть. Далее персонаж попадает на первый уровень,который ему надо пройти. Для того, чтобы пройти игру, нужно пройти все уровни. Изначально для прохождения доступен только первый уровень, а по мере прохождения открываются последующие уровни. 
Процесс прохождения уровней заключается в преодолении препятствий,зарабатыванием очков, взаимодействии с усилителями, предметами и локациями игрового мира. Пройдя уровень игрок получает доступ к следующему уровню. Игрок может взаимодействовать с игровым миром с помощью компьютерной мыши и клавиш(Space,Ctrl,Q,E). При получении усилителя игрок может использовать его при помощи клавиши Q,при получении определенного кол-ва очков игрок может получить доп.жизнь клавишей E,перемещение осуществляется автоматически кроме прыжков и подкатов(Space и Ctrl). Доступны следующие объекты взаимодействия с игроком:
\begin{enumerate}
\item Усилитель – при нажатии клавиши Q активирует усилитель. При активации усилителя  придает ГГ ускорение на некоторое время.
\item Кольца – предметы являющиеся обычными очками набираемыми при прохождении. Если у ГГ находится больше 50 колец, при помощи клавиши E можно активировать доп.жизнь.
\item Переход на другую локацию – при наведении иконка курсора меняется на стрелку в направлении доступной локации, в которую хочет перейти пользователь.
\end{enumerate}

\subsection{Сюжет игры}

Главный герой безымянный рыцарь.............................

\subsection{Интерфейс пользователя}

На основании анализа предметной области в программе должны быть реализованы следующие прецеденты:
\begin{enumerate}
	\item Перемещение по уровням;
	\item Графическое отображение уровней;
	\item Взаимодействие с игровыми объектами при помощи компьютерной мыши и клавиш компьютера.
\end{enumerate}

\subsection{Требования к оформлению документации}

Разработка программной документации и программного изделия должна производиться согласно ГОСТ 19.102-77 и ГОСТ 34.601-90. Единая система программной документации.
