\section{Техническое задание}
\subsection{Основание для разработки}

Основанием для разработки является задание на выпускную квалификационную работу бакалавра "<\Тема \ТемаВтораяСтрока">.

\subsection{Цель и назначение разработки}

Основной задачей выпускной квалификационной работы является разработка платформы для создания игр-платформеров с целью продвижения их популярности. Данный программный продукт предназначен для демонстрации практических навыков, полученных в течение обучения.

Задачами данной разработки являются:
\begin{itemize}
\item анализ существующих игр-платформеров;
\item разработка концептуальной модели игр-платформеров;
\item проектирование программной системы для создания игр-платформеров;
\item реализация программной системы для создания игр-платформеров;
\item тестирование разработанной системы.
\end{itemize}

\subsection{Функции программной системы для создания игр-платформеров}

Программная система должна включать в себя следующие функции:
\begin{itemize}

    \item добавление нового состояния;
    \item смена состояний;
    \item назначение сценариев на состояния;
    \item добавление объекта;
    \item удаление объекта;
    \item удаление всех игровых объектов;
    \item отрисовка игровых объектов;
    \item отрисовка текста;
    \item получения состояния клавиш(нажата или нет);
    \item получение списка объектов заданного типа;
    \item обработка столкновений.
\end{itemize}

\subsection{Описание программной системы для создания игр-платформеров на примере игры}

Для создания новой игры требуется создать класс игры, который наследуется от класса platformer.game.Game. Этот класс является главной частью программной системы.

Любая игра состоит из состояний.В начале мы определяем все состояния игры, например:
\begin{itemize}
	\item заставка,
	\item меню,
	\item игра,
	\item конец игры.
\end{itemize}

Состояние создается с помощью метода new\_state (см. \ref{state:imgage}) в котором мы указываем ссылки на два сценария. Первый сценарий выполняется однократно при переходе на данное состояние. Второй сценарий выполняется в цикле все время пока установлено данное состояние. Например:

\begin{figure}[H]
	\begin{lstlisting}[language=Python]
	class JumpGame(Game):
		'''
		Игра, демонстрирующая движок
		'''
		def __init__(self):
			'''
			Инициализация игры, создание состояний
			'''
			super().__init__(WIDTH, HEIGHT, "Jump game", FPS)
			self.new_state("Intro", self.init_intro, self.intro)
			self.set_state("Intro")
			self.new_state("Game", self.init_game, self.game_script)
			self.new_state("End game", self.end_game, self.start_new_game)
	\end{lstlisting}
	\caption{Пример создания нового сценария}
	\label{state:imgage}
\end{figure}



\subsection{Пример игры}

Игра-платформер моделирует основные механики игр платформеров, в которых игрок управляет персонажем, который передвигается по платформам вверх, собирая монеты и стараясь забраться как можно дальше. Уровень визуализируется в двухмерном виде сбоку с использованием экранной графики персонажей и управляется клавишами. Уровень каждый раз имеет новое расположение платформ и монет.
?Процесс прохождения уровней заключается в преодолении препятствий,зарабатыванием очков, взаимодействии с усилителями, предметами и локациями игрового мира. Пройдя уровень игрок получает доступ к следующему уровню. Игрок может взаимодействовать с игровым миром с помощью компьютерной мыши и клавиш(Space,Ctrl,Q,E). При получении усилителя игрок может использовать его при помощи клавиши Q,при получении определенного кол-ва очков игрок может получить доп.жизнь клавишей E,перемещение осуществляется автоматически кроме прыжков и подкатов(Space и Ctrl). Доступны следующие объекты взаимодействия с игроком:


Создаётся окно Jump Game, в котором отображается  заставка игры, при нажатии клавиши Space заставка сменяется на окно игры, на нем пользователь видит текущий уровень, персонажа, а так же все объекты, находящиеся в нем. Пользователь может взаимодействовать с окном с помощью клавиш компьютера.!!!! Левым кликом мыши по окну вызывает метод mouse\_click у текущей игры. который вызывает проверку находится ли в координатах, в которых был совершён клик, какой-либо персонаж или объект, и если есть, то вызвать метод on\_click. Если персонажа в данных координатах нет, то вызвать у всех персонажей с полем category == "pc" метод search\_position(x,y), который указывает координаты движения, которые должны прийти персонажи. Так же работают все сценарии, конкретной зоны. они работают до тех пор, пока не будет вызвано условие останавливающее, конкретный сценарий.


\subsection{Требования к оформлению документации}

Разработка программной документации и программного изделия должна производиться согласно ГОСТ 19.102-77 и ГОСТ 34.601-90. Единая система программной документации.
Программная документация должна включать в себя:
\begin{itemize}
	\item техническое задание;
	\item технический проект;
	\item рабочий проект.
\end{itemize}