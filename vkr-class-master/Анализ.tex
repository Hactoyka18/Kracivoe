\section{Анализ предметной области}
\subsection{История и описание игр-платформеров}

Платфо́рмер (англ. platformer, platform game) — жанр компьютерных игр, в которых основу игрового процесса составляют прыжки по платформам, лазанье по лестницам, сбор предметов, необходимых для победы над врагами или завершения уровня.Многие игры подобного жанра характеризуются нереалистичностью, рисованной мультяшной графикой. Персонажами таких игр часто бывают вымышленные существа (к примеру, драконы, гоблины) или антропоморфные животные.
Платформеры появились в начале 1980-х и стали трёхмерными ближе к концу 1990-х. Через некоторое время после образования жанра у него появилось данное название, отражающее тот факт, что в платформерах геймплей сфокусирован на прыжках по платформам в противовес стрельбе. Правда, во многих платформерах присутствует стрелковое оружие, в таких, например, как Blackthorne или Castlevania. Последняя послужила основой поджанра метроидвания. 

Некоторые предметы, называемые пауэр-апами (англ. power-up), наделяют управляемого игроком персонажа особой силой, которая обычно иссякает со временем (к примеру: силовое поле, ускорение, увеличение высоты прыжков). Коллекционные предметы, оружие и «пауэр-ап» собираются обычно простым прикосновением персонажа и для применения не требуют специальных действий со стороны игрока. Реже предметы собираются в «инвентарь» героя и применяются специальной командой (такое поведение более характерно для аркадных головоломок). Сходный жанр компьютерных игр сайд-скроллер.

Противники (называемые «врагами»), всегда многочисленные и разнородные, обладают примитивным искусственным интеллектом, стремясь максимально приблизиться к игроку, либо не обладают им вовсе, перемещаясь по круговой дистанции или совершая повторяющиеся действия. Соприкосновение с противником обычно отнимает жизненные силы у героя или вовсе убивает его. Иногда противник может быть нейтрализован либо прыжком ему на голову, либо из оружия, если им обладает герой. Смерть живых существ обычно изображается упрощённо или символически (существо исчезает или проваливается вниз за пределы экрана).

Уровни, как правило, изобилуют секретами (скрытые проходы в стенах, высокие или труднодоступные места), нахождение которых существенно облегчает прохождение и подогревает интерес игрока.
\subsection{История жанра}
Платформеры появились в начале 1980-х, когда игровые консоли не были достаточно мощными, чтобы отображать трёхмерную графику или видео. Они были ограничены статическими игровыми мирами, которые помещались на один экран, а игровой герой был виден в профиль. Персонаж лазал вверх и вниз по лестницам или прыгал с платформы на платформу, часто сражаясь с противниками и собирая предметы, улучшающие характеристики (так называемые «пауэрапы»). Первыми играми этого типа были Space Panic и Apple Panic. За ними последовала игра Donkey Kong, аркадная игра созданная фирмой «Nintendo» и выпущенная в 1981 году. Вскоре процесс прохождения уровня перестал быть в основном вертикальным и стал горизонтальным с появлением длинных многоэкранных прокручивающихся игровых миров. Считается, что начало этому положила выпущенная фирмой Activision в 1982 году игра Pitfall! для консолей Atari 2600. Manic Miner (1983) и её продолжение Jet Set Willy (1984) были наиболее популярными платформерами на домашних компьютерах.

В 1985 году фирма «Nintendo» выпустила для приставки Nintendo Entertainment System революционный платформер Super Mario Bros. Игра была наполнена большими и сложными уровнями, и стала примером для последующих создателей игр, и даже сегодня многие люди считают её одной из самых лучших видеоигр. Игра имела фантастическую популярность и продавалась огромными тиражами. Для многих людей она стала первым в их жизни платформером, а главный герой Марио стал символом фирмы «Nintendo».

Термин «трёхмерный платформер» может обозначать или геймплей, включающий все три измерения, или использование трёхмерных полигонов в реальном времени для отрисовки уровней и героев, или и то и другое. Появление трёхмерных платформеров принесло изменение конечных целей некоторых платформеров. В большинстве двумерных платформеров игроку нужно было достичь на уровне только одной цели, однако во многих трёхмерных платформерах, каждый уровень необходимо прочесывать, собирая кусочки головоломок (Banjo-Kazooie) или звезды (Super Mario 64). Это дало возможность более эффективного использования больших трёхмерных областей и вознаграждало игрока за тщательное исследование уровня, но некоторые игроки считают собирание бесчисленных безделушек более нудным занятием, чем игровые испытания. Donkey Kong 64 была раскритикована за то, что игроку приходилось часто переключаться между пятью различными игровыми героями, чтобы получить бананы различных цветов и другие предметы. Однако не все трёхмерные платформеры были такими, самым ярким примером является Crash Bandicoot. Эта игра оставалось верной традиции двумерных платформеров и в ней использовались довольно плоские уровни, в конце которых располагалась игровая цель.

В 2000-х гг. продолжилось развитие и популяризация 3D-платформеров (Pane. 2016). В 2002 г. была выпущена «Ratchet \& Clank», в которой пользователи наблюдали за игровым процессом как от первого, так и от третьего лица. Также они могли использовать множество видов оружия для борьбы с противниками и преодоления препятствий. В 2005 г. вышла игра «Psychonauts». В процессе прохождения пользователи открывали сверхъестественные психические способности, такие как телекинез, левитация, невидимость и пирокинез. Это позволяло протагонисту исследовать большую часть локаций, а также сражаться с врагами. Иной геймплей был у «Little Big Planet» (2008). В игре, представляющей собой приключенческий платформер, содержались элементы головоломки. Кроме того, пользователям открывалась уникальная возможность редактирования и создания собственных уровней.

В 2010 г. вышла «Super Meat Boy», выполненная в стиле классических 2D-платформеров. В 2015 г. была выпущена «Ori and the Blind Forest», а в 2017 г. «Hollow Knight» и «Cuphead», обладающие уникальным графическим стилем и отличающиеся повышенной сложностью прохождения. В то же время вышли продолжения серий, зародившихся ещё в 1980–1990-х гг. (Pane. 2016). Например, в 2013 г. был выпущен музыкальный 2D-платформер «Rayman Legends», а в 2017 г. «Super Mario Odyssey», выполненный в 3D-графике. Также стали появляться бесконечные платформеры (англ. «endless runner»), предназначенные для мобильных устройств, например, «Temple Run» (2011) и «Jetpack Joyride» (2011). В поздние 2010-е гг. популярность обрели приключенческие платформеры, относящиеся к поджанрам roguelike и метроидвания. К играм данного типа относятся «Dead Cells» 2018 г. и «Blasphemous» 2019 г. Другим направлением развития жанра платформера стали кооперативные 3D-проекты, например «Biped» (2020) и «It Takes Two» (2021).

Как и другие жанры, платформер активно изучают специалисты Game Studies. Особый интерес для учёных представляет игровой опыт пользователей и их восприятие цифровых продуктов. Например функция, которая подразумевает смерть и последующее возрождение подконтрольного персонажа. Данная игровая особенность, свойственная платформерам, тесно связана с формированием впечатлений при ознакомлении пользователей с цифровым продуктом. Так, когда игроки погружены в игровой процесс, риск смерти способен вызвать беспокойство за персонажа. Это побуждает пользователей к взвешенному принятию решений. Сочувствие к персонажу может вызывать сильные эмоции, активизирующиеся при прохождении всё более сложных уровней (Mecler. 2020).

Также учёные исследуют методы создания и совершенствования проектов, например, построение уровней при помощи искусственного интеллекта. К примеру, в 2010 г. команда американских учёных изучала различные подходы генерации уровней 2D-платформеров. Выяснилось, что алгоритм, основанный на ритмах, обеспечивает большую гибкость при создании игровых этапов. Под ритмами понималась комбинация геометрического строения уровня и действий игроков, направленных на прохождение этапа или преодоление возникших препятствий (Launchpad. 2011). Кроме того, учёные рассматривали построение уровней при помощи модульного расширения существующего игрового движка «Unity». Исследователи задействовали структурные (типы и положения платформ) и функциональные (физику и возможности перемещения персонажа) особенности уровней, а также вероятность успеха при совершении одиночных прыжков (Aramini. 2018). В основе такого типа публикаций зачастую лежит статистика игроков, отражающая способы и паттерны прохождения ими уровней.