\addcontentsline{toc}{section}{СПИСОК ИСПОЛЬЗОВАННЫХ ИСТОЧНИКОВ}

\begin{thebibliography}{9}

	\bibitem{pyyavu}Шелудько, В. М. Основы программирования на языке высокого уровня Python: учебное пособие / В. М. Шелудько. – Ростов-на-Дону, Таганрог: Издательство Южного федерального университета, 2017. – 146 c. – ISBN 978-5-9275-2649-9. – Текст: электронный // Электронно-библиотечная система IPR BOOKS: [сайт]. – URL: http://www.iprbookshop.ru/87461.html. – Режим доступа: для авторизир. пользователей
    \bibitem{interface} Мандел, Т. Разработка пользовательского интерфейса : учебное пособие / Т. Мандел. – ДМК Пресс, 2019. – 420 с. – ISBN 978-5-04-195060-6. – Текст : непосредственный.
    \bibitem{Cos} Томакова Р.А.  Методы и алгоритмы цифровой обработки изображений : учебное пособие / Р. А. Томакова, Е. А. Петрик ; Юго-Зап. гос. ун-т. - Курск : Университетская книга, 2020. - 310 с. - Библиогр.: с. 297-309. - ISBN 978-5-907270-19-0. - Текст : непосредственный.
    \bibitem{py1} Чаплыгин А.А. Программирование на языке Python : учебное пособие / Юго-Зап. гос. ун-т ; сост. А. А. Чаплыгин. - Электрон. текстовые дан. (229 КБ). - Курск : ЮЗГУ, 2021. - 15 с. - Загл. с титул. экрана. - Б. ц. - Текст : электронный.
    \bibitem{py2} Северенс, Ч.   Введение в программирование на Python : учебник / Ч. Северенс. - 2-е изд., испр. - Москва : Национальный Открытый Университет «ИНТУИТ», 2016. - 231 с. - URL: https://biblioclub.ru/index.php?page=book\&id=429184 (дата обращения: 24.08.2023) . - Б. ц. - Текст : электронный.
	\bibitem{py3} Хахаев, И. А.   Практикум по алгоритмизации и программированию на Python: курс : учебное пособие / И. А. Хахаев. - 2-е изд., исправ. - Москва : Национальный Открытый Университет «ИНТУИТ», 2016. - 179 с. - URL: https://biblioclub.ru/index.php?page=book\&id=429256 (дата обращения: 24.08.2023) . - Библиогр. в кн. - Б. ц. - Текст : электронный.
	\bibitem{pyyavu} Шелудько, В. М.  Язык программирования высокого уровня Python: функции, структуры данных, дополнительные модули : учебное пособие / В. М. Шелудько ; Министерство науки и высшего образования РФ ; Федеральное государственное автономное образовательное учреждение высшего образования «Южный федеральный университет» ; Институт компьютерных технологий и информационной безопасности. - Ростов-на-Дону|Таганрог : Издательство Южного федерального университета, 2017. - 108 с. : ил. - URL: http://biblioclub.ru/index.php?page=book\&id=500060 (дата обращения: 24.08.2023) . - Режим доступа: по подписке. - Библиогр. в кн. - ISBN 978-5-9275-2648-2 : Б. ц. - Текст : электронный.
	\bibitem{naukaodata} Келлехер, Д.    Наука о данных: базовый курс : учебное пособие / Д. Келлехер, Б. Тирни ; науч. ред. З. Мамедьяров ; пер. с англ. М. Белоголовский. - Москва : Альпина Паблишер, 2020. - 224 с. - URL: http://biblioclub.ru/index.php?page=book\&id=598235 (дата обращения: 09.09.2022) . - Режим доступа: по подписке. - ISBN 978-5-9614-3170-4 : Б. ц. - Текст : электронный.
	\bibitem{serebrovsk} Математические методы и инновационные научно-технические разработки : сборник научных трудов / Федеральное государственное бюджетное образовательное учреждение высшего профессионального образования "Юго-Западный государственный университет" ; редкол.: В. В. Серебровский (отв. ред.) [и др.]. - Курск : ЮЗГУ, 2014. - 282 с. ; 20. - Библиогр. в конце ст. - 100 экз. - ISBN 978-5-7681-0930-1 : 380.00 р. - Текст : непосредственный.
	\bibitem{allfill} Программирование, тестирование, проектирование, нейросети, технологии аппаратно‐программных средств (практические задания и способы их решения) : учебник / С. В. Веретехина, К. С. Кармицкий, Д. Д. Лукашин [и др.]. - Москва : Директ-Медиа, 2022. - 144 с. - URL: https://biblioclub.ru/index.php?page=book\&id=694782 (дата обращения: 10.01.2023) . - Режим доступа: по подписке. - Библиогр. в кн. - ISBN 978-5-4499-3321-8 : Б. ц. - Текст : электронный..    
	\bibitem{intsysandtech} Интеллектуальные системы и технологии : учебное пособие / С. П. Ющенко [и др.] ; Юго-Зап. гос. ун-т. - Курск : Университетская книга, 2018. - 226 с. : ил. - Библиогр.: с. 238-240 (32 назв.). - ISBN 978-5-907138-22-3 : 560.00 р. - Текст : непосредственный.    
	\bibitem{} Системная инженерия. Принципы и практика = Systems engineering principles and practice : учебник / А. Косяков [и др.] ; пер. с англ. под ред. В. К. Батоврин. - 2-е изд. - Москва : ДМК Пресс, 2014. - 624 с. : ил. - Указ.: с. 610-619. - 400 экз. - ISBN 978-5-97060-122-8 (в пер.). - Текст : непосредственный.    
	\bibitem{} Кузнецов, А. С.    Теория вычислительных процессов  : учебник / А. С. Кузнецов, Р. Ю. Царев, А. Н. Князьков. - Красноярск : Сибирский федеральный университет, 2015. - 184 с. - URL: http://biblioclub.ru/index.php?page=book\&id=435696. - ISBN 978-5-7638-3193-1 : Б. ц.  - Текст : электронный.    
	\bibitem{} Исакова, А. И.    Основы информационных технологий : учебное пособие / А. И. Исакова. - Томск : ТУСУР, 2016. - 206 с. : ил. - URL: http://biblioclub.ru/index.php?page=book\&id=480808 (дата обращения: 18.02.2022) . - Режим доступа: по подписке. - Библиогр.: с. 197-198. - Б. ц. - Текст : электронный.
	\bibitem{} Гунько, А. В.    Программирование (в среде Windows) : учебное пособие / А. В. Гунько ; Новосибирский государственный технический университет. - Новосибирск : Новосибирский государственный технический университет, 2019. - 155 с. - URL: https://biblioclub.ru/index.php?page=book\&id=575417 (дата обращения: 03.05.2024) . - Режим доступа: по подписке. - Библиогр. в кн. - ISBN 978-5-7782-3890-9 : Б. ц. - Текст : электронный.
	\bibitem{} Сазонов C.Ю.    Системный подход к моделированию процессов возникновения и развития пожаров : монография / С. Ю. Сазонов ; Юго-Зап. гос. ун-т. - Курск : Деловая полиграфия, 2016. - 218 с. - Библиогр.: с. 213-219. - ISBN 978-5-9907910-9-1. - Текст : непосредственный.
	\bibitem{oopanalyz} Зайцев, М. Г. Объектно-ориентированный анализ и программирование : учебное пособие / М. Г. Зайцев. – Новосибирск : изд-во НГТУ, 2017. – 84 с. – ISBN 978-5-04-112962-0. – Текст : непосредственный.
	\bibitem{uml2} Джеймс, Р. UML 2.0. Объектно-ориентированное моделирование и разработка : практическое пособие / Р. Джеймс, Б. Майкл. – 2-е изд. – Санкт-Петербург : Питер, 2021. – 542 с. – ISBN 978-5-4461-9428-5. – Текст : непосредственный.
	\bibitem{advancedpython} Бейдер, Д. Python Tricks: A Buffet of Awesome Python Features : учебное пособие / Д. Бейдер. – Москва : ДМК Пресс, 2021. – 300 с. – ISBN 978-5-97060-999-7. – Текст : непосредственный.
	\bibitem{fuzzysystems} Клейн, Р. Нечеткие системы в Python : учебное пособие / Р. Клейн. – Москва : ДМК Пресс, 2020. – 320 с. – ISBN 978-5-97060-758-0. – Текст : непосредственный.
	\bibitem{keras} Чоллет, Ф. Глубокое обучение на Python : учебное пособие : в 5 томах / Ф. Чоллет. – Москва : ДМК Пресс, 2018. – 304 с. – ISBN 978-5-97060-409-1. – Текст : непосредственный.
	\bibitem{pythonmachinelearning} Рашка, С., Мирджалили, В. Python и машинное обучение : практическое пособие / С. Рашка, В. Мирджалили. – Москва : ДМК Пресс, 2018. – 418 с. – ISBN 978-5-97060-310-0. – Текст : непосредственный.
	\bibitem{python} Лутц, М. Изучаем Python : учебное пособие  / М. Лутц. –  5-е издание – Санкт-Петербург : Питер, 2019. – 1584 с. – ISBN 978-5-4461-0705-9. – Текст : непосредственный.
	\bibitem{python} Изучаем Python. Программирование игр, визуализация данных, веб-приложения / Э. Мэтиз. – Санкт-Петербург : Питер, 2016. – 544 с. – ISBN 978-5-496-02305-4. – Текст~: непосредственный.
	\bibitem{python} Автоматизация рутинных задач с помощью Python / Э. Свейгарт. – Москва : И.Д. Вильямс, 2016. – 592 с. – ISBN 978-5-8459-20902-4. – Текст~: непосредственный.
	\bibitem{python}Эл Свейгарт: Учим Python, делая крутые игры / Э. Свейгарт. – Москва~:  Бомбора, 2021 г. – 416 с. – ISBN 978-5-699-99572-1. – Текст~: непосредственный.
	\bibitem{python}Программист-прагматик. Путь от подмастерья к мастеру  / Э. Хант, Д. Томас. – Санкт-Петербург : Диалектика', 2020. – 368 с. – ISBN 978-5-907203-32-7. – Текст~: непосредственный.
	\bibitem{python}Совершенный код / С. Макконнелл. – Москва~: Издательство «Русская редакция», 2010. — 896 стр. – ISBN 978-5-7502-0064-1. – Текст~: непосредственный.
	\bibitem{python}Приемы объектно-ориентированного проектирования. Паттерны проектирования / Э. Гамма, Р. Хелм, Р. Джонсон, Дж. Влиссидес. – Санкт-Петербург : Питер, 2021. – 368 с. – ISBN 5-272-00355-1. – Текст~: непосредственный.
	\bibitem{python}Рефакторинг. Улучшение существующего кода / Ф. Мартин. – Москва~: Диалектика-Вильямс, 2019 – 448 с. – ISBN 978-5-9909445-1-0. – Текст~: непосредственный.
	\bibitem{python}Роберт Мартин: Чистый код. Создание, анализ и рефакторинг / Р. Мартин. – Санкт-Петербург~: Питер, 2020 г, 2016 – 464 с. – ISBN 978-5-4461-0960-9. – Текст~: непосредственный.
	\bibitem{python}Федоров, Д. Ю.  Программирование на языке высокого уровня Python : учебное пособие для прикладного бакалавриата / Д. Ю. Федоров. – 2-е изд., перераб. и доп. – Москва : Издательство Юрайт, 2019. – 161 с. – (Бакалавр. Прикладной курс). – ISBN 978-5-534-10971-9. – Текст: электронный // ЭБС Юрайт [сайт]. – URL: https://urait.ru/bcode/437489.
	\bibitem{}Сузи, Р. А. Язык программирования Python : учебное пособие / Р. А. Сузи. - 2-е изд., испр. - Москва : Интернет-Университет Информационных Технологий (ИНТУИТ)|Бином. Лаборатория знаний, 2007. - 327 с. - (Основы информационных технологий). - URL: https://biblioclub.ru/index.php?page=book\&id=233288 (дата обращения: 24.08.2023) . - Режим доступа: по подписке. - ISBN 978-5-9556-0109-0 : Б. ц. - Текст : электронный.
	\bibitem{}Sweigart, A. Разработка компьютерных игр на языке Python : курс лекций / A. Sweigart. - 2-е изд., испр. - Москва : Национальный Открытый Университет «ИНТУИТ», 2016. - 505 с. - URL: https://biblioclub.ru/index.php?page=book\&id=429009 (дата обращения: 24.08.2023) . - Режим доступа: по подписке. - Б. ц. - Текст : электронный.
	\bibitem{}Проектирование программного продукта на Python. Совместная разработка проекта с использованием Git, PyCharm и GitFlic : методические указания по выполнению лабораторной работы для слушателей программы профессиональной переподготовки «Программирование на Python» / Юго-Зап. гос. ун-т ; сост.: В. В. Ефремов [и др.]. - Электрон. текстовые дан. (9 173 КБ). - Курск : ЮЗГУ, 2023. - 67 с. - Загл. с титул. экрана. - Б. ц. - Текст : электронный.
\end{thebibliography}
