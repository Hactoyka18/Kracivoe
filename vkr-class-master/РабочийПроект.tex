\section{Рабочий проект}
\subsection{Классы, используемые при разработке сайта}

\subsubsection{Класс Game}

Класс Game относится к platformer и используется для управления программной системы.

Описание методов класса Game представлено в таблице \ref{table:game_methods}.
\renewcommand{\arraystretch}{0.8} % уменьшение расстояний до сетки таблицы
\begin{xltabular}{\textwidth}{|X|X|}
	\caption{Методы класса Game}\label{table:game_methods} \\
	\hline \centrow
	Название метода & \centrow  Описание метода \\
	\hline \centrow 1 & \centrow 2 \\ \hline
	\endfirsthead
	\continuecaption{Продолжение таблицы \ref{table:game_methods}}\\
	\hline \centrow 1 & \centrow 2 \\ \hline
	\finishhead
	add\_object(self, obj) & Добавляет объект obj в игру,параметр obj: добавляемый объект. \\
	\hline
	remove\_object(self, obj) & Удаляет объект obj из игры, параметр obj: удаляемый объект. \\
	\hline
	clear(self) & Удаляет все игровые объекты. \\
	\hline
	key\_pressed(self, key) & Возвращает True если была нажата клавиша key, иначе False, параметр key: нажимаемая кнопка. \\
	\hline
	new\_state(self, name, first\_func, func) & Создание нового игрового состояния, параметр name: имя состояния, параметр first\_func:функция, которая запускается при переключении в это игровое состояние, параметр func: функция, которая работает каждый кадр в этом игровом состянии. \\
	\hline
	set\_state(self, name) & Переключиться в игровое состояние, параметр name: имя состояния. \\
	\hline
	run(self) & Главный цикл игры. \\
	\hline
	render\_text(self, text) & Отрисовка текстового объекта, параметр text: текстовый объект. \\
	\hline
	get\_collision(self, obj, obj\_type) & Получение списка объектов столкновения, параметр obj: объект, с которым проверяется столкновение, параметр obj\_type:тип объектов, которые будут проверяться. \\
	\hline
	get\_objects(self, obj\_types) & Получение списка объектов заданного типа, параметр obj\_types: список требуемых типов. \\
	\hline
	start\_script(self, script\_function, script\_name, *args) & Запускает сценарий в отдельном потоке с возможностью остановки и передачи аргументов, параметр script\_function: функция, содержащая код сценария, параметр script\_name: имя сценария, параметр args: дополнительные аргументы, которые передаются в сценарий. \\
	\hline
	stop\_script(self, script\_name) & Останавливает сценарий по имени, параметр script\_name: имя сценария, который нужно остановить. \\
	\hline
\end{xltabular}



Можно выделить следующий список классов и их методов, использованных при разработке web-приложения (таблица \ref{class:table}). Пример таблицы с уменьшенным межстрочным интервалом.

\renewcommand{\arraystretch}{0.8} % уменьшение расстояний до сетки таблицы
\begin{xltabular}{\textwidth}{|X|p{2.5cm}|>{\setlength{\baselineskip}{0.7\baselineskip}}p{4.85cm}|>{\setlength{\baselineskip}{0.7\baselineskip}}p{4.85cm}|}
\caption{Описание классов Bitrix, используемых в приложении\label{class:table}}\\
\hline \centrow \setlength{\baselineskip}{0.7\baselineskip} Название класса & \centrow \setlength{\baselineskip}{0.7\baselineskip} Модуль, к которому относится класс & \centrow Описание класса & \centrow Методы \\
\hline \centrow 1 & \centrow 2 & \centrow 3 & \centrow 4\\ \hline
\endfirsthead
\caption*{Продолжение таблицы \ref{class:table}}\\
\hline \centrow 1 & \centrow 2 & \centrow 3 & \centrow 4\\ \hline
\finishhead
CMain & Главный модуль & CMain – главный класс страницы web-приложения. После одного из этапов по загрузке страницы в сценарии становится доступным инициализированный системой объект данного класса с именем \$APPLICATION & void ShowTitle(string property\_code = «title», bool strip\_tags = true)
Выводит заголовок страницы
void SetTitle(string title)
Устанавливает заголовок страницы

void ShowCSS(bool external = true, bool XhtmlStyle = true)
Выводит таблицу стилей CSS страницы\\
\hline CFile & Главный модуль & CFile – Класс для работы с файлами и изображениями & array GetFileArray (int file\_id)
Метод возвращает массив, содержащий описание файла (путь к файлу, имя файла, размер) с идентификатором file\_id
\end{xltabular}
\renewcommand{\arraystretch}{1.0} % восстановление сетки

\subsection{Модульное тестирование разработанного web-сайта}

Модульный тест для класса User из модели данных представлен на рисунке \ref{unitUser:image}.

\begin{figure}[ht]
\begin{lstlisting}[language=Python]
from django.test import TestCase
from .models import *
User = get_user_model()


class ShpoTestCases(TestCase):

    def setUp(self) -> None:
        self.user = User.objects.create(username='testtestovich', password='testtestovich', first_name='Sad', last_name='')

    def test_2(self):

        self.assertEqual(self.user.first_name, 'Sad')
        self.assertEqual(self.user.last_name, 'Cat')
        print((self.user))
        print((self.user.first_name))
        print((self.user.last_name))
\end{lstlisting}  
\caption{Модульный тест класса User}
\label{unitUser:image}
\end{figure}

\subsection{Системное тестирование разработанного web-сайта}

На рисунке \ref{main:image} представлена главная страница сайта «Русатом – Аддитивные технологии».
\newpage % при необходимости можно переносить рисунок на новую страницу
\begin{figure}[H] % H - рисунок обязательно здесь, или переносится, оставляя пустоту
\center{\includegraphics[width=1\linewidth]{main1}}
\center{\includegraphics[width=1\linewidth]{main2}}
\center{\includegraphics[width=1\linewidth]{main3}}
\caption{Главная страница сайта «Русатом – Аддитивные технологии»}
\label{main:image}
\end{figure}

На рисунке \ref{menu:image} представлен динамический вывод заголовков, включающий в себя искомые фразы при поиске фраз.

\begin{figure}[ht]
\center{\includegraphics[width=1\linewidth]{menu}}
\caption{Динамический вывод заголовков}
\label{menu:image}
\end{figure}

На рисунке \ref{enter:image} представлен ввод данных для публикации новости.

\begin{figure}[ht]
\center{\includegraphics[width=1\linewidth]{enter}}
\caption{Ввод данных для публикации очень-очень длинной, интересной и полезной новости}
\label{enter:image}
\end{figure}
